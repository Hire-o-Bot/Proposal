\section{Feasibility Analysis}
\subsection{Technical Feasibility}
The feasibility analysis for our project dictates that its technical requirements can be fulfilled. It includes suitable tech stacks that are available as well as technical expertise. 
\begin{itemize}
    \item Algorithms are available to tackle the machine learning requirements. 
    \item Network models exist that will help evaluate the performance of candidates using NLP. 
    \item Our solution will be web-based, and there are stacks available to back our solution on the web. The details of the stacks we will be using is given below:
    \begin{itemize}
        \item Javascript
        \item React
        \item Node
        \item Express
        \item MongoDB
        \item Amazon S3
        \item Microsoft Azure
        \item Python
        \item SCSS
        \item Tailwind CSS
        \item SCSS
    \end{itemize}
    \item Since we may handle real-time video data, cloud-based MicroServices can be utilized to keep the performance figures in check. 
    \item Our team includes individuals with diverse skill sets having proficient knowledge in Web Development, AI, ML, CVIP, and solution architecture.
\end{itemize}

Each of the techonology mentioned above are available free of cost and the technical skills are easily manageable. Time limitations and implementation using these technologies are well synchronized. In the development phase, our project will be hosted on a free hosting website but one's it is in production, we will host it on a paid hosting with enough bandwidth. All the resources are easily available. Solution can be easily implemented using these availabe resources so from there it is cleared that the project "Automating Recruiting Workflow to Hire Candidates" is technically feasible.

\subsection{Operational Feasibility}
Depending on the present condition of the industry, our project will be readily accepted as almost every organization is in need of it. The conventional method requires a series of repetitive steps that are very time-consuming and most of all they are very expensive. On the other hand, our solution will help trim down the HR department since this whole procedure is automated and can be supervised by a maximum of two employees. In this it is more feasible as the cost required to maintain our solution is exponentially less than using conventional hiring procedure. Lastly, a lot of time will be saved by automation and in a professional environment time is one of the most important assets which can then be used for some other project making it more efficient.

\subsection{Economical Feasibility}
Our solution tends to be more cost-effective than the present procedure as it reduces the number of employees required to manage HR work while also increasing productivity and efficient use of resources i.e time. In the conventional method, we repeat certain steps for recruitment which in our system are automated thus reducing the amount of effort. On the other hand, the spare time that we get by automating these steps can be used in other projects thus organization workflow becomes more efficient. Our system requires a maximum of 2 employees to operate. However, we need a full panel of recruiters for the conventional method thus proving that our solution is far more feasible than the present system.

\subsection{Time Feasibility}
The timeline of your project will go here. (Leave this section blank). 
